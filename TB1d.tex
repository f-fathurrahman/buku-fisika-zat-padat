\chapter{Model Tight Binding 1d}

Asumsi:

\begin{itemize}
\item Terdapat satu orbital pada atom $n$, yaitu $\ket{n}$
\item Sistem memiliki syarat batas periodik.
\item Semua orbital saling ortogonal: $\braket{n}{m} = \delta_{nm}$
\end{itemize}

Asumsikan fungsi gelombang berbentuk:
\[
\ket{\Psi} = \sum_{n} \phi_{n} \ket{n}
\]

    Persamaan Schroedinger: \[
\sum_{m} H_{nm} \phi_{m} = E \phi_{n}
\] dengan elemen matriks Hamiltonian: \[
H_{nm} = \bra{n}H\ket{m}
\]

    Hamiltonian dapat dituliskan sebagai: \[
H = K + \sum_{j} V_{j}
\] dengan $K = \mathbf{p}^2/(2m)$ adalah energi kinetik dan $V_{j}$
adalah interaksi Coulomb antara elektron pada posisi $\mathbf{r}$
dengan inti atom pada situs $j$: \[
V_{j} = V(\mathbf{r} - \mathbf{R}_{j})
\] dengan $\mathbf{R}_{j}$ adalah posisi dari inti ke-$j$.

    Dengan definisi tersebut, kita memiliki: \[
H\ket{m} = (K + V_{m})\ket{m} + \sum_{j \neq m} V_{j}\ket{m}
\]

    $K + V_{m}$ adalah Hamiltonian yang diperoleh jika hanya ada satu
atom, yaitu atom ke-$m$, pada sistem: \[
(K + V_{m})\ket{m} = \epsilon_{\mathrm{a}}\ket{m}
\] di mana $\epsilon_{a}$ adalah energi elektron pada atom ke-$m$
jika tidak ada atom lain.

    \[
H_{n,m} = \bra{n}H\ket{m} = \epsilon_{a}\delta_{n,m} + \sum_{j \neq m} \bra{n}V_{j}\ket{m}
\]

    Suku penjumlahan potensial dapat diaproksimasi menjadi: \[
\sum_{j \neq m} \bra{n}V_{j}\ket{m} = \begin{cases}
V_{0} & n = m \\
-t & n = m \pm 1 \\
0 & \text{lainnya}
\end{cases}
\]

    Matriks Hamiltonian menjadi: \[
H_{nm} = \epsilon_{0}\delta_{nm} - t(\delta_{n+1,m} + \delta_{n-1,m})
\] di mana $\epsilon_{0} = \epsilon_{a} + V_{0}$.

    Parameter $t$ dikenal juga sebagai hopping term. Suku ini menjadi
besar apabila orbital semakin dekat dan turun secara eksponensial jika
orbital semakin jauh.

    Dengan menggunakan ansatz: \[
\phi_{n}(k) = \frac{e^{-\imath k n a}}{\sqrt{N}}
\]

    Diperoleh: \[
\sum_{m} H_{nm}\phi_{m} = \epsilon_{0}\frac{e^{-\imath k n a}}{\sqrt{N}} -
t\left(
\frac{e^{-\imath k (n+1) a}}{\sqrt{N}} +
\frac{e^{-\imath k (n-1) a}}{\sqrt{N}}
\right)
\] dan: \[
E\phi_{n} = E \frac{e^{-\imath k n a}}{\sqrt{N}}
\]

\[
E \frac{e^{-\imath k n a}}{\sqrt{N}} = 
\epsilon_{0}\frac{e^{-\imath k n a}}{\sqrt{N}} -
t\left(
\frac{e^{-\imath k (n+1) a}}{\sqrt{N}} +
\frac{e^{-\imath k (n-1) a}}{\sqrt{N}}
\right)
\]

Energi: \[
E(k) = \epsilon_{0} - 2t\cos(ka)
\] dispersion relation, band structure
