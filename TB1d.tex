\chapter{Model Tight Binding 1d}

\section{Model monoatomik}

{\centering
\includegraphics{images_priv/Simon_Fig_11_1.pdf}
\par}

Tinjau suatu model rantai tight-binding 1d monoatomik.
Misalkan jarak antara atom adalah $a$.
Orbital pada atom $n$ dilambangkan dengan $\ket{n}$ untuk
$n = 1,\ldots,N$.

Asumsi:
\begin{itemize}
\item Terdapat satu orbital pada atom $n$, yaitu $\ket{n}$
\item Sistem memenuhi syarat batas periodik.
\item Semua orbital saling ortogonal: $\braket{n | m} = \delta_{nm}$
\end{itemize}

Asumsikan fungsi gelombang berbentuk:
\begin{equation*}
\ket{\Psi} = \sum_{n} \phi_{n} \ket{n}
\end{equation*}

Persamaan Schroedinger:
\begin{equation*}
\sum_{m} H_{nm} \phi_{m} = E \phi_{n}
\end{equation*}
dengan elemen matriks Hamiltonian:
\begin{equation*}
H_{nm} = \bra{n}H\ket{m}
\end{equation*}

Hamiltonian dapat dituliskan sebagai:
\begin{equation*}
H = K + \sum_{j} V_{j}
\end{equation*}
dengan $K = \mathbf{p}^2/(2m)$ adalah energi kinetik dan $V_{j}$
adalah interaksi Coulomb antara elektron pada posisi $\mathbf{r}$
dengan inti atom pada situs $j$:
\begin{equation*}
V_{j} = V(\mathbf{r} - \mathbf{R}_{j})    
\end{equation*}
dengan $\mathbf{R}_{j}$ adalah posisi dari inti ke-$j$.

Dengan definisi tersebut, kita memiliki:
\begin{equation*}
H\ket{m} = (K + V_{m})\ket{m} + \sum_{j \neq m} V_{j}\ket{m}    
\end{equation*}

$K + V_{m}$ adalah Hamiltonian yang diperoleh jika hanya ada satu
atom, yaitu atom ke-$m$, pada sistem:
\begin{equation*}
(K + V_{m})\ket{m} = \epsilon_{\mathrm{a}}\ket{m}    
\end{equation*}
di mana $\epsilon_{a}$ adalah energi elektron pada atom ke-$m$
jika tidak ada atom lain.
\begin{equation*}
H_{n,m} = \bra{n}H\ket{m} = \epsilon_{a}\delta_{n,m} + \sum_{j \neq m} \bra{n}V_{j}\ket{m}
\end{equation*}

Suku penjumlahan potensial dapat diaproksimasi menjadi:
\begin{equation*}
\sum_{j \neq m} \bra{n}V_{j}\ket{m} = \begin{cases}
V_{0} & n = m \\
-t & n = m \pm 1 \\
0 & \text{lainnya}
\end{cases}
\end{equation*}

Matriks Hamiltonian menjadi:
\begin{equation*}
H_{nm} = \epsilon_{0}\delta_{nm} - t(\delta_{n+1,m} + \delta_{n-1,m})
\end{equation*}
di mana $\epsilon_{0} = \epsilon_{a} + V_{0}$.
Parameter $\epsilon_{0}$ dikenal juga dengan nama energi \textit{on-site}.
Sedangkan parameter $t$ dikenal juga dengan nama $hopping term$.
Parameter ini menjadi
besar apabila orbital semakin dekat dan turun secara eksponensial jika
orbital semakin jauh.

Persamaan Schroedinger yang akan diselesaikan memiliki bentuk:
\begin{equation*}
\sum_{m} \left[
  \epsilon_{0}\delta_{nm} - t(\delta_{n+1,m} + \delta_{n-1,m})
\right] \phi_{m} =
E \phi_{n}
\end{equation*}
atau:
\begin{equation*}
\epsilon_{0} \phi_{n} - t \left( \phi_{n+1} + \phi_{n-1} \right) =
E \phi_{n}
\end{equation*}

Untuk mencari solusi dari persamaan tersebut, kita akan menggunakan ansatz:
\begin{equation*}
\phi_{n}(k) = e^{-\imath k n a}
\end{equation*}
Substitusi ansatz ke persamaan Schroedinger, diperoleh:
\begin{equation*}
\epsilon_{0} e^{-\imath k n a} - t\left(
e^{-\imath k (n+1) a} + e^{-\imath k (n-1) a}
\right) = E e^{-\imath k n a}
\end{equation*}
atau:
\begin{equation*}
E = \epsilon_{0} - t\left(
e^{-\imath k a} + e^{\imath k a} \right)
\end{equation*}
Kita mendapatkan energi elektron sebagai fungsi dari vektor gelombang $k$
\begin{equation*}
E(k) = \epsilon_{0} - 2t\cos(ka)
\end{equation*}
Hubungan ini juga merupakan suatu hubungan dispersi yang lebih dikenal
dengan nama struktur pita elektronik.

Mirip dengan kasus vibrasi 1d, untuk sistem periodik, nilai $k$ adalah diskrit:
\begin{equation*}
k = \frac{2\pi p}{Na}
\end{equation*}
di mana $p$ adalah bilangan bulat dan $Na$ adalah panjang dari sistem.


\section{Model TB 1d Diatomik}

Sekarang kita akan meninjau sistem yang mirip dengan sebelumnya
namun sekarang diasumsikan bahwa terdapat dua jenis atom
pada sistem:
\begin{equation*}
- A - B - A - B - A - B -
\end{equation*}
Energi \textit{on-site} dari atom $A$ berbeda dengan atom $B$.
Dengan menggunakan asumsi nearest-neighbor interaction, kita mendapatkan:
\begin{align*}
E \phi_{n}^{A} = \epsilon_{A} \phi_{n}^{A} - t \left( \phi_{n}^{B} + \phi_{n-1}^{B} \right) \\
E \phi_{n}^{B} = \epsilon_{B} \phi_{n}^{B} - t \left( \phi_{n}^{A} + \phi_{n+1}^{A} \right) \\
\end{align*}
Dengan menggunakan ansatz:
\begin{align*}
\phi_{n}^{A} = Ae^{\imath k n a}
\phi_{n}^{B} = Be^{\imath k n a}
\end{align*}
Diperoleh:
\begin{align*}
EA = \epsilon_{A} A - t ( 1 + e^{-\imath k a}) B \\
EB = \epsilon_{A} B - t ( 1 + e^{-\imath k a}) A
\end{align*}

Persamaan ini adalah persamaan eigen. Nilai eigen $E$ dapat
dicari dengan cara mencari akar dari determinan:
\begin{equation*}
\begin{vmatrix}
\epsilon_{A} - E & -t(1 + e^{\imath k a}) \\
-t(1 + e^{\imath k a}) & \epsilon_{B} - E \\
\end{vmatrix}
\end{equation*}
Diperoleh persamaan sekuler:
\begin{equation*}
E^2 - E( \epsilon_A + \epsilon_B ) + \epsilon_{A} \epsilon_{B} -
t^2 (1 + 1 + e^{\imath ka} + e^{-\imath ka}) = 0
\end{equation*}
dengan solusi:
\begin{equation*}
E_{\pm}(k) = \frac{1}{2}\left(
\epsilon_{A} + \epsilon_{B} \pm \sqrt{
(\epsilon_{A} - \epsilon_{B})^2 + 4t^2(2 + 2\cos(ka))
}
\right)
\end{equation*}
