\chapter{Vibrasi Padatan}
\section{Model rantai atom 1d}

Tinjau suatu rantai atom 1d yang terdiri dari partikel atau atom identik
dengan massa \(m\) dengan jarak ekuilibrium antar atom adalah \(a\)
(yang juga merupakan konstanta kisi dari sistem).

Posisi dari atom ke-\(n\) dinyatakan sebagai \(x_{n}\) dan posisi
ekuilibriumnya sebagai \(x^{0}_{n} = na\).

Atom akan bergerak atau bervibrasi disekitar posisi ekuilibrium dan
definisikan \[
\delta x_{n} = x_{n} - x^{0}_{n}
\]

Jika sistem berada pada temperatur yang cukup rendah, kita dapat
mengasumsikan bahwa potensial yang bekerja pada atom merupakan potensial
harmonik, yang lambangkan dengan pegas yang menghubungkan antara atom.
Oleh karena itu, model ini juga dikenal sebagai model rantai harmonik.

Energi potensial antar atom:
\begin{align*}
V_{\mathrm{tot}} & = \sum_{i} \frac{\kappa}{2} \left(
x_{i+1} - x_{i} - a \right)^2 \\
& = \sum_{i} \frac{\kappa}{2} \left(
\delta x_{i+1} - \delta x_{i} \right)^2
\end{align*}

Gaya yang bekerja pada massa ke-\(n\) dari rantai atom adalah (asumsi
nearest neighbor interaction):
\begin{align*}
F_{n} & = -\frac{\partial V_{\mathrm{tot}}}{\partial x_{n}} \\
& = \kappa \left( \delta x_{n+1} - \delta x_{n} \right) +
\kappa \left( \delta x_{n-1} - \delta x_{n} \right)
\end{align*}

Persamaan Newton:
\begin{equation*}
m \ddot{\delta x_{n}} = F_{n} = \kappa \left(
    \delta x_{n+1} + \delta x_{n-1} - 2\delta x_{n}
    \right)    
\end{equation*}

Asumsikan solusi berbentuk:
\begin{equation*}
\delta x_{n} = A e^{\imath \omega t - \imath k x^{0}_{n}} = 
A e^{\imath \omega t - \imath k n a}
\end{equation*}
dengan $A$ adalah amplitudo osilasi, $k$ adalah vektor
gelombang (bilangan gelombang), dan $\omega$ adalah frekuensi
gelombang.

Substitusi ke persamaan diferensial:
\begin{equation*}
-m\omega^2 A e^{\imath \omega t - \imath k n a} = 
\kappa A e^{\imath \omega t} \left[ 
e^{-\imath k (n+1) a} +
e^{-\imath k (n-1) a} +
2e^{-\imath k n a}
\right]
\end{equation*}

Sederhanakan:
\begin{equation*}
m\omega^2 = 2\kappa\left( 1 - \cos(ka) \right) = 4\kappa \sin^2 \left( \frac{ka}{2} \right)
\end{equation*}
Menggunakan identitas trigonometri: $\cos(2A) = 1 - 2\sin^2(A)$
Diperoleh (hubungan dispersi):
\begin{equation}
\omega(k) = 2\sqrt{\frac{\kappa}{m}} \left| \sin\left(\frac{ka}{2}\right) \right|
\end{equation}

$\omega(k)$: hubungan dispersi

\section{Ruang resiprokal pada 1d}

Hubungan dispersi merupakan fungsi yang periodik dalam \(k\) dengan
periode \(2\pi/a\). Hal ini merupakan prinsip umum: \emph{Suatu sistem
yang periodik pada ruang nyata dengan periode \(a\) akan periodik pada
ruang resiprokal dengan periode \(2\pi/a\)}

Unit sel pada ruang resiprokal atau ruang-\(k\) dikenal dengan nama zona
Brillouin. Zona Brillouin pertama adalah unit sel pada ruang-\(k\) yang
memiliki pusat pada titik \(k=0\).

Titik \(k=\pm \pi/a\) dikenal sebagai batas zona Brillouin.

Brillouin zone: $-\pi/a \leq k < \pi/a$

Mode vibrasi telah didefinisikan memiliki bentuk: \[
\delta x_{n} = A e^{\imath \omega t - \imath k n a}
\]

Jika kita mengganti \(k\) dengan \(k + 2\pi/a\), diperoleh:
\begin{align}
\delta x_{n} & = A e^{\imath \omega t - \imath (k + 2\pi/a) n a} \\
& = A e^{\imath \omega t - \imath k n a} e^{-\imath 2\pi n} \\
& = A e^{\imath \omega t - \imath k n a}
\end{align}


Menggunakan \[
e^{-\imath 2\pi n } = \cos(2\pi n) - \sin(2\pi n) = 1
\] untuk sembarang bilangan bulat \(n\).

Kita menemukan bahwa menggeser \(k \rightarrow k + 2\pi/a\) memberikan
kembali mode osilasi awal.

Secara umum menggeser bilangan gelombang \(k\) dengan
\(k \rightarrow 2\pi p/a\), dengan merupakan bilangan bulat, akan
memberikan kita mode vibrasi awal.

Kita dapat mendefinisikan himpunan titik pada ruang-\(k\) yang secara
fisis ekuivalen dengan titik \(k=0\). Himpunan titik ini dikenal sebagai
kisi resiprokal (\emph{reciprocal lattice}).

Sedangkan himpunan titik \(x_{n} = na\) disebut sebagai kisi langsung
(\emph{direct lattice}) atau kisi ruang nyata (\emph{real space
lattice}).

Kita dapat melihat analogi antara kisi langsung dan kisi reciprokal:
\begin{align}
x_{n} & = \cdots, -2a, -a, 0, a, 2a, \cdots \\
G_{n} & = \cdots, -2\left(\frac{2\pi}{a}\right), -\left(\frac{2\pi}{a}\right), 0, \left(\frac{2\pi}{a}\right), 2\left(\frac{2\pi}{a}\right), \cdots
\end{align}


Secara definisi, hubungan antara kisi langsung dan kisi resiprokal
adalah: \[
e^{\imath G_{m} x_{n}} = 1
\]

\section{Properti dari dispersi rantai atom 1d}

\subsection{Gelombang suara}

\[
k = \frac{2\pi}{\lambda}
\]

\[
v = \frac{\omega}{k}
\]

\[
\omega = 2 \pi f
\]

\[
v = \lambda f
\]

Gelombang suara adalah vibrasi yang memiliki panjang gelombang yang
besar dibandingkan dengan jarak antar atom. Pada daerah ini, kita
menemukan bahwa hubungan dispersi yang kita hitung memiliki hubungan
linear terhadap vektor gelombang, yaitu:
\(\omega = v_{\mathrm{sound}} k\) dengan: \[
v_{\mathrm{sound}} = a\sqrt{\frac{\kappa}{m}}
\]

Untuk nilai-nilai \(k\) yang lebih besar, hubungan dispersi tidak lagi
linear. Hal ini tidak sejalan dengan apa yang diasumsikan oleh Debye.

Pada panjang gelombang yang pendek atau \(k\) yang besar, kita
mendefinisikan dua kecepatan yang berbeda: kecepatan grup dan kecepatan
fasa.

Kecepatan grup, yaitu kecepatan pergerakan paket gelombang yang
diberikan oleh: \[
v_{\mathrm{group}} = \frac{\mathrm{d}\omega}{\mathrm{d}k}
\]

Kecepatan fasa adalah kecepatan pergerakan satu komponen frekuensi dari
gelombang \[
v_{\mathrm{phase}} = \frac{\omega}{k}
\]

Kedua kecepatan ini bernilai sama untuk hubungan dispersi linear, namun
secara umum kedua kecepatan tersebut tidak sama.

Kecepatan grup menjadi nol pada batas zona Brillouin, yaitu pada
\(k = \pm \pi/a\).

\subsection{Menghitung jumlah mode normal}

Asumsikan sistem memiliki \(N\) partikel pada rantai dan asumsikan
syarat batas periodik, yaitu partikel \(x_{0}\) berbatasan dengan
\(x_{1}\) disebelah kanan dan \(x_{N-1}\) di bagian kiri, atau \[
x_{n+N} = x_{n}
\]

Kita harus memiliki: \[
e^{\imath \omega t - \imath k n a} =
e^{\imath \omega t - \imath k (N+n) a}
\] atau \[
e^{\imath k N a} = 1
\]

\[
cos(kNa) + \imath \sin(kNa) = 1
\]

\[
\cos(kNa) = 1
\]

\[
\sin(kNa) = 0
\]

Persyaratan ini memberikan batasan pada nilai-nilai \(k\) yang mungkin:
\[
k = \frac{2\pi p}{Na} = \frac{2\pi p}{L}
\] di mana \(p\) adalah bilangan bulat dan \(L\) adalah panjang total
dari sistem.

Dengan kata lain: \(k\) menjadi terkuantisasi.

Sekarang, kita akan menghitung ada berapa banyak mode normal yang kita
miliki. Karena periodisitas dari ruang-\(k\), kita hanya perlu
memperhitungkan nilai \(k\) yang berada pada zona Brillouin pertama
yaitu: \(-\pi/a \leq k < \pi/a\).

Jumlah total mode normal adalah jangkauan nilai \(k\) dibagi dengan
jarak antara \(k\) yang berdekatan: \[
\text{jumlah mode normal} = \frac{2\pi/a}{2\pi/(Na)} = N
\]

Artinya: terdapat satu mode normal untuk tiap massa pada sistem, atau
\emph{satu mode normal per satu derajat kebebasan pada seluruh sistem}.

Debye telah memprediksikan hal ini dengan menggunakan frekuensi cutoff
(frekuensi Debye) pada model vibrasi padatan Debye.

Foton: \(\hbar \nu\), fonon: \(\hbar \omega\)

\section{Mode Kuantum: fonon}

Korespondensi kuantum: jika suatu sistem harmonik klasik memiliki mode
osilasi normal pada frekuensi \(\omega\) maka sistem kuantum yang
berkorespondensi akan memiliki keadaan eigen dengan energi: \[
E_{n} = \hbar \omega \left( n + \frac{1}{2} \right)
\]

Pada vektor gelombang \(k\) yang diberikan, terdapat banyak keadaan
eigen yang mungkin, dengan keadaan dasar pada \(n=0\) yang hanya
memiliki energi titik-nol (\emph{zero point energy}) sebesar
\(\hbar \omega(k)/2\).

Eksitasi energi terendah (\(n=1\)) memiliki energi lebih besar
\(\hbar\omega(k)\) dibandingkan dengan keadaan dasar.

Secara umum, semua eksitasi pada vektor gelombang ini terjadi dalam
satuan energi \(\hbar\omega(k)\) dan energi yang lebih tinggi
berkorespondensi secara klasi dengan osilasi yang memiliki amplitudo
lebih besar.

Tiap eksitasi dari mode normal ini dengan langkah \(\hbar\omega(k)\)
dikenal sebagai \textbf{fonon}.

Fonon dapat didefinisikan sebagai kuanta vibrasi diskrit.

Hal ini mirip dengan mendefinisikan kuanta cahaya sebaga foton.
Sebagaimana dengan foton, kita dapat memperlakukan fonon sebagai suatu
partikel atau sebagai gelombang terkuantisasi.

Jika kita memperlakukan fonon sebagai partikel, kita dapat mempersiapkan
banyak fonon pada satu keadaan yang sama (atau bilangan kuantum \(n\)
pada persamaan energi osilator harmonik dapat memiliki sembarang nilai).

Dengan kata lain fonon, seperti foton, adalah boson. Pada temperatur
bukan nol, akan terdapat fonon yang menempati suatu mode tertentu (atau
\(n\) secara rata-rata akan lebih besar dari 0) berdasarkan faktor
okupansi Bose: \[
n_{B}(\beta\hbar\omega) = \frac{1}{e^{\beta\hbar\omega} - 1}
\] dengan \(\beta = 1/(k_{B}T)\) dan \(\omega\) adalah frekuensi osilasi
dari mode.

Nilai ekspektasi energi dari fonon pada vektor gelombang \(k\) diberikan
oleh: \[
E_{k} = \hbar\omega(k)\left(
n_{B}(\beta\hbar\omega(k)) + \frac{1}{2}
\right)
\]

Kita dapat menggunakan persamaan ini untuk menghitung kapasitas panas
dari model rantai atom 1d.

Total energi adalah: \[
U_{\mathrm{total}} = \sum_{k} \hbar \omega(k) \left(
n_{B}(\beta\hbar\omega(k)) + \frac{1}{2}
\right)
\] di mana penjumlahan terhadap \(k\) dilakukan untuk semua mode normal
yang mungkin, yaitu \(k = 2\pi p /(Na)\) sedemikian rupa sehingga
\(-\pi/a \leq k < \pi/a\)

Dalam ekspresi lain:
\begin{equation*}
\sum_{k} \rightarrow \sum_{p = -N/2}^{(N/2)-1}  
\end{equation*}

Untuk sistem yang besar, nilai-nilai \(k\) sangat berdekatan, sehingga
kita dapat mengubah tanda penjumlahan diskrit menjadi integral, sehingga
diperoleh: \[
\sum_{k} \rightarrow \frac{Na}{2\pi} \int_{-\pi/a}^{\pi/a}\mathrm{d}k
\]

Kita dapat menggunakan integral ini untuk menghitung jumlah mode pada
sistem: \[
\frac{Na}{2\pi} \int_{-\pi/a}^{\pi/a}\mathrm{d}k = N
\]

Dengan menggunakan bentuk integral, total energi dapat ditulis sebagai:
\[
U_{\mathrm{total}} = \frac{Na}{2\pi}
\int_{-\pi/a}^{\pi/a} \mathrm{d}k \, \hbar \omega(k) \left(
n_{B}(\beta\hbar\omega(k)) + \frac{1}{2}
\right)
\]

Persamaan ini mirip dengan apa yang digunakan oleh Debye untuk
menghitung kapasitas kalor, \(\mathrm{d}U/\mathrm{d}T\), jika model yang
digunakan adalah 1d. Perbedaannya adalah \(\omega(k)\) yang digunakan
oleh Debye adalah linear: \(\omega = vk\).

Model Einstein untuk perhitungan kapasitas kalor juga dapat diturunkan,
yaitu untuk kasus di mana frekuensi \(\omega\) konstan untuk seluruh
\(k\).

\section{Model Rantai Diatomik 1d}


\begin{align}
m \ddot{\delta x_{n}} = \kappa_{2} ( \delta y_{n} - \delta x_{n} ) +
\kappa_{1} ( \delta y_{n-1} - \delta x_{n} ) \\
m \ddot{\delta y_{n}} = \kappa_{1} ( \delta x_{n+1} - \delta y_{n} ) +
\kappa_{2} ( \delta x_{n} - \delta y_{n} )
\end{align}


\begin{align}
\delta x_{n} & = A_{x} e^{\imath \omega t - \imath kna} \\
\delta y_{n} & = A_{y} e^{\imath \omega t - \imath kna}
\end{align}


\begin{align}
-\omega^2 m A_{x} e^{\imath \omega t - ikna} & =
\kappa_{2} A_{y} e^{\imath \omega t - ikna} +
\kappa_{1} A_{y} e^{\imath \omega t - ik(n-1)a} -
(\kappa_{1} + \kappa_{2}) A_{x} e^{\imath \omega t - ikna} \\
-\omega^2 m A_{y} e^{\imath \omega t - ikna} & =
\kappa_{1} A_{x} e^{\imath \omega t - ik(n+1)a} +
\kappa_{2} A_{x} e^{\imath \omega t - ikna} -
(\kappa_{1} + \kappa_{2}) A_{y} e^{\imath \omega t - ikna}
\end{align}


\begin{align}
-\omega^2 m A_{x} & = \kappa_{2} A_{y} + \kappa_{1} A_{y} e^{\imath ka} -
(\kappa_{1} + \kappa_{2})A_{x} \\
-\omega^2 m A_{y} & = \kappa_{1} A_{x} e^{-\imath k a} + \kappa_{2} A_{x} -
(\kappa_{1} + \kappa_{2})A_{y}
\end{align}

\begin{equation*}
m\omega^2 \begin{bmatrix}
A_{x} \\
A_{y}
\end{bmatrix} = 
\begin{bmatrix}
(\kappa_{1} + \kappa_{2}) & -\kappa_{2} - \kappa_{1} e^{\imath k a} \\
(-\kappa_{2} - \kappa_{1})e^{-\imath k a} & (\kappa_{1} + \kappa_{2})
\end{bmatrix}
\begin{bmatrix}
A_{x} \\
A_{y}
\end{bmatrix}    
\end{equation*}


Persamaan eigenvalue: \(\mathbf{A} \mathbf{x} = \lambda \mathbf{x}\)

\begin{align}
\begin{vmatrix}
(\kappa_{1} + \kappa_{2}) - m\omega^2 & -\kappa_{2} - \kappa_{1} e^{\imath k a} \\
(-\kappa_{2} - \kappa_{1})e^{-\imath k a} & (\kappa_{1} + \kappa_{2}) - m\omega^2
\end{vmatrix} & = 0 \\
\left| (\kappa_{1} + \kappa_{2}) - m\omega^2 \right|^2 -
\left| \kappa_{2} + \kappa_{1}e^{\imath k a} \right|^2 & = 0
\end{align}

\begin{equation*}
m\omega^2 = (\kappa_{1} + \kappa_{2}) \pm 
\left| \kappa_{1} + \kappa_{2} e^{\imath k a}\right|    
\end{equation*}

\begin{align}
\left| \kappa_{1} + \kappa_{2} e^{\imath ka} \right| & = 
\sqrt{( \kappa_{1} + \kappa_{2} e^{\imath ka} )
  ( \kappa_{1} + \kappa_{2} e^{-\imath ka} ) } \\
& = \sqrt{ \kappa_{1}^{2} + \kappa_{2}^{2} + 2\kappa_{1} \kappa_{2} \cos(ka) }
\end{align}


\begin{align}
\omega_{\pm} & = \sqrt{\frac{\kappa_{1} + \kappa_{2}}{m} \pm
\frac{1}{m}\sqrt{ \kappa_{1}^{2} + \kappa_{2}^{2} + 2\kappa_{1} \kappa_{2} \cos(ka) }
} \\
& = \sqrt{\frac{\kappa_{1} + \kappa_{2}}{m} \pm
\frac{1}{m}\sqrt{ (\kappa_{1} + \kappa_{2})^{2} - 4\kappa_{1} \kappa_{2} \sin^{2}(ka/2) }
}
\end{align}

Untuk tiap \(k\) kita mendapatkan dua mode normal, yang biasanya disebut
sebagai dua cabang dispersi.

Karena terdapat \(N\) nilai \(k\) yang berbeda, kita mendapatkan \(2N\)
total mode. Hal ini sesuai dengan intuisi Debye bahwa untuk setiap
derajat kebebasan pada sistem kita memiliki satu mode normal.

Terdapat cabang dispersi dengan hubungan yang mendekati linear ketika
\(k \rightarrow 0\). Cabang ini adalah mode akustik.

Cabang dispersi dengan energi yang lebih tinggi dikenal sebagai mode
optik. Pada mode akustik, frekuensi menjadi
\(\sqrt{2(\kappa_{1} + \kappa_{2})/m}\) pada \(k = 0\).
