\section{Model Einstein alternatif}

Padatan dianggap sebagai kumpulan osilator harmonik.
Energi eigen dari satu osilator harmonik:
\begin{equation}
E_{n} = \hbar \omega ( n + 1/2 )
\end{equation}
dengan $\omega$ adalah frekuensi dari osilator harmonik, atau
frekuensi Einsten.
Fungsi partisi dari sistem ini adalah:
\begin{align*}
Z_{1D} & = \sum_{n \geq 0} e^{-\beta\hbar\omega(n + 1/2)} \\
& = \frac{e^{-\beta\hbar\omega/2}}{1 - e^{-\beta\hbar\omega}}
\end{align*}
Nilai ekspektasi dari energi adalah:
\begin{align*}
\braket{E} & = -\frac{1}{Z_{1D}}\frac{\partial Z_{1D}}{\partial \beta} \\
& = \hbar \omega \left( n_{B}(\beta\hbar\omega) + \frac{1}{2} \right)
\end{align*}
dengan $n_{B}$ adalah faktor okupansi Bose:
\begin{equation}
n_{B}(x) = \frac{1}{e^{x} - 1}
\end{equation}
Kapasitas kalor dapat dihitung dari:
\begin{equation}
C = \frac{\partial \braket{E}}{\partial T} = k_{B} (\beta \hbar \omega )^2
\frac{e^{\beta\hbar\omega}}{(e^{\beta\hbar\omega} - 1)^2}
\end{equation}
Pada limit temperatur yang tinggi, kita mendapatkan $C = k_{B}$.

Untuk kasus 3D kita memiliki:
\begin{equation*}
E_{n_x,n_y,n_z} = \hbar \omega \left[
\left(n_x + \frac{1}{2}\right) +
\left(n_y + \frac{1}{2}\right) +
\left(n_z + \frac{1}{2}\right)
\right]
\end{equation*}
dan fungsi partisi:
\begin{equation*}
Z_{3D} = \sum_{n_x,n_y,n_z \geq 0} e^{\beta E_{n_x,n_y,n_z}} = (Z_{1D})^{3}
\end{equation*}
sehingga:
\begin{equation*}
\braket{E_{3D}} = 3\braket{E_{1D}}
\end{equation*}
dan
\begin{equation*}
C = 3 k_{B} (\beta \hbar \omega )^2
\frac{e^{\beta\hbar\omega}}{(e^{\beta\hbar\omega} - 1)^2}
\end{equation*}

Pada limit temperatur yang tinggi, atau $k_{B}T >> \hbar \omega$ kita
mendapatkan $C = 3k_{B}$
