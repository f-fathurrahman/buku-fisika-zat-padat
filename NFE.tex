\chapter{Elektron pada Padatan}

\subsection{Nearly-free electron model}

Tinjau elektron bebas dengan Hamiltonian:
\begin{equation}
H_{0} = \frac{\mathbf{p}^2}{2m}
\end{equation}
Keadaan eigen adalah gelombang bidang $\mathbf{k}$ dengan energi eigen
\begin{equation}
\epsilon_{0}(\mathbf{k}) = \frac{\hbar^2 \mathbf{k}^2}{2m}
\end{equation}

Anggap sekarang Hamiltonian diberikan potensial perturbasi periodik:
\begin{equation}
H = H_{0} + V(\mathbf{r})
\end{equation}
Potensial $V$ adalah periodik dengan periode $\mathbf{R}$:
\begin{equation}
V(\mathbf{r}) = V(\mathbf{r} + \mathbf{R})
\end{equation}
Elemen matriks untuk potensial ini adalah:
\begin{equation}
\braket{\mathbf{k}' | V | \mathbf{k} } =
\frac{1}{L^3} \int \mathrm{d}\mathbf{r}\, e^{\imath(\mathbf{k} - \mathbf{k}')\cdot
\mathbf{r}} \, V(\mathbf{r}) \equiv V_{\mathbf{k}' - \mathbf{k}}
\end{equation}
Integral ini akan bernilai nol, kecuali jika $\mathbf{k}' - \mathbf{k}$ adalah
vektor kisi resiprokal.

Kita akan menggunakan teori perturbasi untuk mencari pengaruh perturbasi terhadap
energi eigen.
Untuk koreksi orde-1 kita mendapatkan:
\begin{equation}
\epsilon_{k} = \epsilon_{0}(\mathbf{k}) + 
\braket{\mathbf{k} | V | \mathbf{k} } =
\epsilon_{0}(\mathbf{k}) + V_{0}
\end{equation}
Hal ini berarti energi eigen hanya digeser dengan suatu energi konstan.
Konstanta ini akan diasumsikan bernilai 0.

Menggunakan koreksi orde-2 diperoleh:
\begin{equation}
\epsilon(\mathbf{k}) = \epsilon_{0}(\mathbf{k}) +
\sum_{\mathbf{k}' = \mathbf{k} + \mathbf{G}}
\frac{\braket{\mathbf{k}' | V | \mathbf{k}}}
{\epsilon_{0}(\mathbf{k}) - \epsilon_{0}(\mathbf{k}')}
\end{equation}
di mana pada penjumlahan kita mensyaratkan $\mathbf{G} \neq \mathbf{0}$.
Kita perlu hati-hati dalam menggunakan koreksi ini, karena untuk beberapa
$\mathbf{k}'$ bisa jadi kita nilai $\epsilon_{0}(\mathbf{k})$
sangat dekat atau sama dengan $\epsilon_{0}(\mathbf{k}')$, atau keadaan degenerasi.
Situasi ini biasanya terjadi pada batas zona Brillouin. Untuk mengatasi hal ini
kita perlu menggunakan teori pertubasi untuk keadaan degenerate.

Jika dua gelombang bidang, misalkan $\ket{\mathbf{k}}$ dan
$\ket{\mathbf{k}'} = \ket{\mathbf{k} + \mathbf{G}}$
memiliki energi yang sama, maka kita harus mendiagonalisasi elemen matriks
dari keadaan tersebut.
Elemen matriks dari dua gelombang bidang tersebut adalah:
\begin{align*}
\braket{\mathbf{k} | H | \mathbf{k}} & = \epsilon_{0}(\mathbf{k}) \\
\braket{\mathbf{k}' | H | \mathbf{k}'} & = \epsilon_{0}(\mathbf{k}') =
\epsilon_{0}(\mathbf{k} + \mathbf{G}) \\
\braket{\mathbf{k} | H | \mathbf{k}'} & = V_{\mathbf{k}-\mathbf{k}'} =
V^{*}_{\mathbf{G}} \\
\braket{\mathbf{k}' | H | \mathbf{k}} & = V_{\mathbf{k}'-\mathbf{k}} =
V_{\mathbf{G}}
\end{align*}

Pada ruang dua dimensi ini, kita dapat menuliskan fungsi gelombang sebagai:
\begin{equation}
\ket{\Psi} = \alpha \ket{\mathbf{k}} + \beta \ket{\mathbf{k}'} =
\alpha \ket{\mathbf{k}} + \beta \ket{\mathbf{k} + \mathbf{G}}
\end{equation}
Dengan menggunakan prinsip variasi, kita mendapatkan persamaan Schroedinger efektif
sebagai berikut.
\begin{equation}
\begin{bmatrix}
\epsilon_{0}(\mathbf{k}) & V_{\mathbf{G}}^{*} \\
V_{\mathbf{G}} & \epsilon_{0}(\mathbf{k} + \mathbf{G})
\end{bmatrix}
\begin{bmatrix} \alpha \\ \beta \end{bmatrix} =
E \begin{bmatrix} \alpha \\ \beta \end{bmatrix}
\end{equation}
atau persamaan karakteristik
\begin{equation}
\left( \epsilon_{0}(\mathbf{k}) - E \right)
\left( \epsilon_{0}(\mathbf{k} + \mathbf{G}) - E \right) - 
V_{\mathbf{G}}^{2} = 0
\end{equation}

Untuk kasus $\mathbf{k}$ tepat berada pada batas zona Brillouin, kita
memiliki $\epsilon_{0}(\mathbf{k}) = \epsilon_{0}(\mathbf{k} + \mathbf{G})$
dan persamaan karakteristik:
\begin{equation}
\left( \epsilon_{0}(\mathbf{k}) - E \right)^2 = V_{\mathbf{G}}^2
\end{equation}
atau:
\begin{equation}
E_{\pm} = \epsilon_{0} \pm V_{\mathbf{G}}
\end{equation}
Kita mendapatkan bahwa terdapat gap pada batas zona Brillouin.

Kita sekarang akan meninjau untuk kasus di mana $\mathbf{k}$
berada di sekitar zona Brillouin.
Untuk memudahkan perhitungan, kita akan fokus pada sistem 1d.
Untuk kasus ini batas zona Brillouin adalah
$k = \pm \pi/a$, dan
$G = \pm 2\pi n /a$.
Tinjau $k = n\pi/a + \delta$ dan $k = -n\pi/a + \delta$.
Substitusi ke persamaan karakteristik sehingga kita memperoleh:
\begin{equation}
\left( \frac{\hbar^2}{2m}\left[ (n\pi/a)^2 + \delta^2 \right] - E \right)^2 =
\left( \frac{\hbar^2}{2m} 2n\pi\delta/a \right)^2 + V_{\mathbf{G}}^2
\end{equation}
dan solusi:
\begin{equation}
E_{\pm} = \frac{\hbar^2}{2m} \left[ (n\pi/a)^2 + \delta^2 \right] \pm
\sqrt{\left( \frac{\hbar^2}{2m} 2n\pi\delta/a \right)^2 + V_{\mathbf{G}}^2}
\end{equation}
Untuk nilai $\delta$ yang kecil, kita memperoleh:
\begin{equation}
E_{\pm} = \frac{\hbar^2}{2m} (n\pi/a)^2 \pm |V_{\mathbf{G}}| +
\frac{\hbar^2 \delta^2}{2m} \left[
1 \pm \frac{\hbar^2 (n\pi/a)^2}{m} \frac{1}{|V_{\mathbf{G}}|}
\right]
\end{equation}
Artinya: di sekitar celah pita pada zona Brillouin hubungan dispersi bersifat kuadratik
dalam $\delta$.

{\centering
\includegraphics[scale=1.5]{images_priv/Simon_Fig_15_3.pdf}
\par}

\section{Teorema Bloch}

Keadaan eigen dari suatu elektron pada potensial periodik memiliki
bentuk sebagai berikut:
\begin{equation}
\Psi^{\alpha}_{\mathbf{k}}(\mathbf{r}) = e^{\imath \mathbf{k} \cdot \mathbf{r} }
u^{\alpha}_{\mathbf{k}}(\mathbf{r})
\end{equation}
di mana $u^{\alpha}_{\mathbf{k}}(\mathbf{k})$ adalah fungsi yang memiliki periodisitas
sama dengan potensial dan sel satuan.
$\alpha$ adalah indeks dari keadaan pada masing-masing $\mathbf{k}$.

Kita dapat menggunakan ekspansi dengan basis gelombang bidang pada untuk
$u^{\alpha}_{\mathbf{k}}$:
\begin{equation}
u^{\alpha}_{\mathbf{k}}(\mathbf{r}) = \sum_{\mathbf{G}}
\tilde{u}^{\alpha}_{\mathbf{G},\mathbf{k}} e^{\imath \mathbf{G} \cdot \mathbf{r}}
\end{equation}
sehingga
\begin{equation}
\Psi^{\alpha}_{\mathbf{k}}(\mathbf{r}) = \sum_{\mathbf{G}}
\tilde{u}^{\alpha}_{\mathbf{G},\mathbf{k}} e^{\imath (\mathbf{G} + \mathbf{k}) \cdot \mathbf{r}}
\end{equation}