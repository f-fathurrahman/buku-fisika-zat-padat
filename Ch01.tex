\chapter{Model Zat Padat tanpa Struktur Mikroskopik}

Hukum Dulong-Petit (1819), kapasitas kalor dari banyak padatan:
\begin{align}
C & = 3k_{B}\,\,\text{per atom} \\
\text{atau}\,\, C & = 3R
\end{align}


\section{Teori kinetika gas}
Asumsi:
\begin{itemize}
\item partikel gas bergerak bebas (random) dalam ruang tertutup
\item partikel gas bisa menumbuk dinding
\item partikel gas berbentuk bola
\item partikel gas bisa saling bertumbukan, tetapi tidak saling berinteraksi
\end{itemize}

Karena partikel terus-menerus bertumbukan secara random, maka rata-rata momentumnya
adalah nol:
\begin{equation}
\braket{p} = 0
\end{equation}
Energi kinetik partikel pada temperatur $T$ adalah
\begin{equation}
E_{k} = \frac{1}{2} m v^2 = k_{B} T
\end{equation}
Misalkan terdapat $N$ partikel dan $\mathrm{d}N$ melambangkan jumlah partikel
gas yang memiliki kecepatan antara $v$ dan $v + \mathrm{d}v$, maka fraksi partikel
dengan kecepatan antara $v$ dan $v + \mathrm{d}v$ adalah $\frac{\mathrm{d}N}{N}$.

Untuk menghitung rata-rata energi partikel kita memerlukan teori ekipartisi. Dalam teori
ekipartisi, semua partikel bebas bergerak sepanjang sumbu $x$, $y$, dan $z$.
Pada ekuilibrium, energi total partikel terbagi rata untuk semua derajat kebebasan
gerakan partikel. Setiap derajat kebebasan dari partikel yang bergerak mempunyai energi
\begin{equation*}
E = \frac{1}{2} m v^2 = \frac{1}{2} k_{B} T
\end{equation*}

Nilai ekspektasi dari energi total adalah:
\begin{align*}
\braket{E} & = E_{x} + E_{y} + E_{z} \\
& = \frac{1}{2} m v_{x}^2 + \frac{1}{2} m v_{y}^2 + \frac{1}{2} m v_{z}^2 \\
& = \frac{3}{2} k_{B} T
\end{align*}

Jika partikel adalah molekul monotaomik, maka gerakan molekul yang terlibat hanya
akibat translasi pada arah $x$, $y$, dan $z$.

Jika partikel adalah molekul diatomi, maka gerakan yang terlibah adalah translasi, rotasi,
dan vibrasi. Gerakan rotasi melibatkan momen inersia $E = \frac{1}{2} I \omega^2$
dan gerakan vibrasi melibatkan energi $E = \frac{1}{2}kx^2$.

Jika ada $N$ partikel pada temperatur $T$ maka energi total adalah
$\braket{E} = \frac{1}{2}Nk_{B}T$.

Dalam teori kinetik gas, energi potensial akibat itneraksi antar partikel gas dianggap nol.

Dari ekspresi energi total kita dapat menghitung kapasitas kalor spesifik sebagai:
\begin{equation*}
C_{v} = \frac{\mathrm{d}\braket{E}}{\mathrm{d}T} = \frac{3}{2}Nk_{B} = \frac{3}{2}R
\end{equation*}
yang merupakan konstanta, tidak bergantung pada temperatur.

Jika partikel gas dimampatkan sehingga menjadi padatan, bagaimana $C_{v}$?

Dulong dan Petit (1819) mengukur beberapa $C_{v}$ dari beberapa padatan dan hasilnya adalah
konstan, yaitu sekitar 6 kalori/mol.derajat.
Dalam eksperimen ini padatan diberikan kalor $Q$ sehingga suhunya naik dari $T_{1}$ ke $T_{2}$.
Kalor jenis dapat dihitung sebagai $C_{v} = \frac{\mathrm{d}Q}{\mathrm{d}T}$.

Ludwig Boltzmann pada 1870, mengusulkan teori untuk menghitung $C_{v}$ dengan asumsi:
\begin{itemize}
\item setiap atom terikat dengan atom lain membentuk osilator harmonik
\item atom bervibrasi ketika mendapat kalor atau dipanaskan
\item setiap atom memiliki energi interaksi $E_{p} = \frac{3}{2}kT$ dan
energi kinetik $E_{k} = \frac{3}{2}kT$
\end{itemize}
Dengan menggunakan teori ekipartisi, nilai ekspektasi energi total adalah:
\begin{equation*}
\braket{E} = E_{p} + E_{k} = 3kT
\end{equation*}
Jika dalam padatan terdapat $N$ ataom, maka total energi adalah
\begin{equation*}
\braket{E} = 3NkT = 3RT
\end{equation*}
sehingga
\begin{equation*}
C_{v} = \frac{\mathrm{d}\braket{E}}{\mathrm{d}T} = 3R
\end{equation*}
atau 6 kalori/(mol derajat).

Teori Boltzmann mendukung hasil eksperimen Dulong-Petit ($T$ tinggi).

Bagaimana nilai $C_{v}$ pada T rendah?

Komerling Onnes(1911) mencairkan gas Nitrogen sehingga kita dapat mencapai
suhu 4.2 K. Berkembang eksperimen pada T rendah, termasuk pengukuran $C_v$
pada T rendah? Hasilnya adalah XXXgrafikXXX.

\section{Model Einstein}

Albert Einstein mengembangkan teori Boltzmann dengan asumsi
\begin{itemize}
\item Padatan terdiri dari $N$ atom, dipandang sebagai $3N$ atom yang direpresentasikan
dengan $3N$ osilator harmonik yang identik dengan frekuensi yang sama, yaitu $\nu$.
\item Osilator adalah osilator kuantum yang memiliki energi diskrit dengan energi
rata-rata:
\begin{equation*}
\braket{E} = \frac{h\nu}{e^{h\nu/kT} - 1}
\end{equation*}
\item Osilator membentuk ensembel yang mengikuti distribusi Maxwell-Boltzmann
\item Setiap osilator memiliki 3 derajat kebebasan sehingga energi perosilator
adalah $3\braket{E}$.
\end{itemize}

Dari asumsi tersebut dapat dihitung energi per satu mol padatan sebagai $E = 3N_{mathrm{A}}\braket{E}$
dengan $N_{\mathrm{A}}$ adalah bilangan Avogadro:
\begin{equation*}
E = 3 N_{A} \frac{h\nu}{e^{h\nu/kT} - 1}
\end{equation*}
Kapasitas kalor:
\begin{align*}
C_{v} & = \left( \frac{\partial E}{\partial T} \right)_{v} \\
& = \frac{\partial}{\partial T}\left(
\frac{h\nu}{e^{h\nu/kT} - 1}
\right) \\
& = 3N_{\mathrm{A}}h\nu
\frac{e^{h\nu/kT} h\nu/kT^2}{(e^{h\nu/kT} - 1)^2} \\
& = 3N_{\mathrm{A}}k \left( \frac{h\nu}{kT} \right)^2
\frac{e^{h\nu/kT}}{(e^{h\nu/kT} - 1)^2}
\end{align*}
atau
\begin{equation}
C_{v} = 3R \left(\frac{\theta_{E}}{T}\right)^2
\frac{e^{\theta_{E}/T}}{(e^{\theta_{E}/T} - 1)^2}
\end{equation}

dengan $R = N_{A}k$ dan $\theta_{E} = \frac{h\nu}{k}$ adalah temperatur Einstein.

Untuk temperatur tinggi $T >> \frac{h\nu}{k}$ atau $T >> \theta_{E}$
\begin{equation*}
e^{h\nu/kT} - 1 = \left( 1 + \frac{h\nu}{kT} + \ldots \right) - 1 = \frac{h\nu}{kT}
\end{equation*}
diperoleh:
\begin{equation*}
\braket{E} = \frac{h\nu}{e^{h\nu/kT} - 1} \approx \frac{h\nu}{h\nu/kT} = kT
\end{equation*}
sehingga:
\begin{equation*}
C_{v} = \left( \frac{\partial E}{\partial T} \right)_{v} = 3R
\end{equation*}
yang sama dengan hasil klasik Dulong-Petit.


Untuk temperatur rendah: $T << \frac{h\nu}{k}$ atau $T << \theta_{E}$
atau $e^{h\nu/kT} >> 1$ sehingga:
\begin{equation*}
\braket{E} = \frac{h\nu}{e^{h\nu/kT} - 1} \approx \frac{h\nu}{e^{h\nu/kT}} =
h\nu e^{-h\nu/kT}
\end{equation*}
Untuk 1 mol padatan diperoleh $E = 3N_{A}\braket{E}$, sehingga:
\begin{align*}
C_{v} & = 3N_{A} h\nu \frac{\partial}{\partial T}\left(
e^{-h\nu/kT} \right) \\
& = 3N_{A} h\nu \frac{h\nu}{kT^2} e^{-h\nu/kT} \\
& = 3N_{A} k \left( \frac{h\nu}{kT} \right)^2 \\
& = 3R \left( \frac{\theta_{E}}{T} \right) e^{\theta_{E}/T}
\end{align*}
Hasil dari model Einstein memberikan  $C_{v} \propto e^{-\theta_{E}/T}$
sedangkan eksperimen memberikan hasil $C_{v} \propto T^{3}$.
Perbedaan ini terjadi karena Einstein berasumsi osilator bervibrasi dengan frekuensi sama.

\section{Model Debye}

Peter Debye (1912) mengembangkan teori Boltzmann dengan asumsi:
\begin{itemize}
\item Padatan terdiri dari $N$ atom, bervibrasi normal mode 3N dengan frekuensi
$\nu_{1}, \nu_{2}, \nu_{3} \ldots \nu_{3N}$
\item Debye menempatkan frekuensi cut-off $\nu_{D}$, dimana pada frekuensi lebih
tinggi dari $\nu_{D}$ tidak ada mode vibrasi padatan,
\item Padatan dipandang sebagai fonon gas (molekul tersusun dalam padatan bervibrasi secara elastis)
\item Fonon sebagai partikel identik dengan spin nol,
tidak dapat dibedakan dan mengikuti distribusi Bose-Einstein.
\item Atom-atom bervibrasi dalam 2 mode: 1 vibrasi longitudinal (searah dengan rambatan) dan 2 vibrasi
transversal (tegak lurus arah rambatan) pada frekuensi antara $\nu$ dan $\nu + \mathrm{d}\nu$
\end{itemize}


\section{Model Einstein alternatif}

Padatan dianggap sebagai kumpulan osilator harmonik.
Energi eigen dari satu osilator harmonik:
\begin{equation}
E_{n} = \hbar \omega ( n + 1/2 )
\end{equation}
dengan $\omega$ adalah frekuensi dari osilator harmonik, atau
frekuensi Einsten.
Fungsi partisi dari sistem ini adalah:
\begin{align*}
Z_{1D} & = \sum_{n \geq 0} e^{-\beta\hbar\omega(n + 1/2)} \\
& = \frac{e^{-\beta\hbar\omega/2}}{1 - e^{-\beta\hbar\omega}}
\end{align*}
Nilai ekspektasi dari energi adalah:
\begin{align*}
\braket{E} & = -\frac{1}{Z_{1D}}\frac{\partial Z_{1D}}{\partial \beta} \\
& = \hbar \omega \left( n_{B}(\beta\hbar\omega) + \frac{1}{2} \right)
\end{align*}
dengan $n_{B}$ adalah faktor okupansi Bose:
\begin{equation}
n_{B}(x) = \frac{1}{e^{x} - 1}
\end{equation}
Kapasitas kalor dapat dihitung dari:
\begin{equation}
C = \frac{\partial \braket{E}}{\partial T} = k_{B} (\beta \hbar \omega )^2
\frac{e^{\beta\hbar\omega}}{(e^{\beta\hbar\omega} - 1)^2}
\end{equation}
Pada limit temperatur yang tinggi, kita mendapatkan $C = k_{B}$.

Untuk kasus 3D kita memiliki:
\begin{equation*}
E_{n_x,n_y,n_z} = \hbar \omega \left[
\left(n_x + \frac{1}{2}\right) +
\left(n_y + \frac{1}{2}\right) +
\left(n_z + \frac{1}{2}\right)
\right]
\end{equation*}
dan fungsi partisi:
\begin{equation*}
Z_{3D} = \sum_{n_x,n_y,n_z \geq 0} e^{\beta E_{n_x,n_y,n_z}} = (Z_{1D})^{3}
\end{equation*}
sehingga:
\begin{equation*}
\braket{E_{3D}} = 3\braket{E_{1D}}
\end{equation*}
dan
\begin{equation*}
C = 3 k_{B} (\beta \hbar \omega )^2
\frac{e^{\beta\hbar\omega}}{(e^{\beta\hbar\omega} - 1)^2}
\end{equation*}

Pada limit temperatur yang tinggi, atau $k_{B}T >> \hbar \omega$ kita
mendapatkan $C = 3k_{B}$



