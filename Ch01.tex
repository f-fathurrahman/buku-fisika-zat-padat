\chapter{Model Zat Padat tanpa Struktur Mikroskopik}

Hukum Dulong-Petit (1819), kapasitas kalor dari banyak padatan:
\begin{align}
C & = 3k_{B}\,\,\text{per atom} \\
\text{atau}\,\, C & = 3R
\end{align}

\section{Model Einstein}

Padatan dianggap sebagai kumpulan osilator harmonik.
Energi eigen dari satu osilator harmonik:
\begin{equation}
E_{n} = \hbar \omega ( n + 1/2 )
\end{equation}
dengan $\omega$ adalah frekuensi dari osilator harmonik, atau
frekuensi Einsten.
Fungsi partisi dari sistem ini adalah:
\begin{align*}
Z_{1D} & = \sum_{n \geq 0} e^{-\beta\hbar\omega(n + 1/2)} \\
& = \frac{e^{-\beta\hbar\omega/2}}{1 - e^{-\beta\hbar\omega}} \\
& = \frac{1}{2\sinh(\beta\hbar\omega/2)}
\end{align*}
Nilai ekspektasi dari energi adalah:
\begin{align*}
\braket{E} = -\frac{1}{Z_{1D}}\frac{\partial Z_{1D}}{\partial \beta}
= \frac{\hbar\omega}{2}
\end{align*}

